\section{Discussion}
\label{sec:discussion}

PINN-based traveltime tomography has clearly shown from the synthetic experiments that it can be used as a reliable tool alternative to conventional tomography. It has a significant advantage over the traditional tool in that it does not need to have a good initial guess which requires knowing a priori information on the investigated area. Nevertheless, the selection of hyperparameters (network architectures,  number of optimization iterations, minibatch size, weights of the loss components) plays an important role in the accuracy of the retrieved model. Among them, special attention needs to be paid to balance the weights of the loss components. I achieved robust convergence in the examples I showed by taking the weight of the data fitting term 100 times more than the PDE loss (eikonal) term. The influence of the weight of the regularizer on the problem is another important factor so that careful consideration needs to be given to decide the acceptable value for the weight of this specific term.  Therefore, optimizing the weights for each component of the loss along with the network weights could be a solution to this issue thus removing the need for time-consuming trial-and-error tasks.