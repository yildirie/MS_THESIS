\chapter{PINN Enabled Traveltime Tomography}
\label{chap:pinn_traveltime}

Seismic first arrival traveltime picking and then inverting them have been used as a common approach for visualizing near-surface seismic velocities~\cite{zsa:92,zt:98,tncc:09}.  Accurately estimating long-wavelength velocity structures in near-surface is essentially important for engineering and environmental geophysical applications. Moreover, reliably retrieved macro-feature velocities can serve as an initial for Full-waveform inversion (FWI) technique ~\cite{t:84,so:85,bszc:95,p:99,roivhd:04,vo:09} to obtain more accurate near-surface velocity models.


Therefore, in this chapter, I describe the PINN-based traveltime inversion as a new tool for retrieving near-surface velocity models and show its effectiveness through synthetic examples by comparing it with the traditional gradient-based inversion.