Sismik araştırmada, yeraltının yapısal ve litolojik bileşimini ortaya çıkarmak için büyük miktarda veri toplanır ve işlenir. Bu prosedürdeki kilit adım, hız modeli oluşturmadır (VMB). İlk varış seyahat zamanının ters çevrilmesi, sismik keşifte yüzeye yakın hız yapılarını tahmin etmek için yaygın olarak kullanılan VMB araçlarından biridir. Veriler (seyahat süreleri) ve model parametreleri (hız) arasındaki bağlantıyı tanımlayan temel matematiksel model, birinci mertebeden doğrusal olmayan kısmi diferansiyel denklem olan eikonal denklemdir. Geleneksel olarak, tersine çevirme, ışın tabanlı yöntemler veya gradyan tabanlı algoritmalar kullanılarak gerçekleştirilir. Gradyan tabanlı algoritmalar, model parametrelerini ışın izleme gerektirmeden güncellemek için gereken gradyanı bulsa da, hesaplama açısından zorlayıcı olabilir. Öte yandan, sağlamlığına ve verimliliğine rağmen ışın tabanlı yöntemler, ışın teorisi yüksek frekanslı yaklaşıma dayandığından karmaşık bölgelerden muzdariptir. Bu yaklaşımları bir seyahat zamanı tersine çevirme problemi için kullanmak yerine, yüzeye yakın yeraltı hızlarını tahmin etmek için eikonal denklem tarafından temsil edilen matematiksel modelden yararlanan fizik bilgili sinir ağlarından (PINN'ler) özellikle yararlanan makine öğrenimi (ML) tabanlı bir yaklaşım öneriyorum. Temel problemi tanımlayan fizik yardımıyla sinir ağlarının eğitilmesi, kabul edilebilir bir ön bilgi gerektirme ve optimizasyonda yanlış parametre güncellemeleri gibi geleneksel yaklaşımların doğasında bulunan bazı zorlukların üstesinden gelir. Sentetik testler ve gerçek bir verinin uygulanması yoluyla, geleneksel tomografi çerçevelerine potansiyel bir alternatif araç olabilecek PINN tabanlı seyahat zamanı inversiyonunun güvenilirliğini gösteriyorum.
	
